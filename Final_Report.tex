
\documentclass{article} % For LaTeX2e
\usepackage[final]{colm2025_conference}

\usepackage{microtype}
\usepackage{hyperref}
\usepackage{url}
\usepackage{booktabs}

\usepackage{lineno}

\definecolor{darkblue}{rgb}{0, 0, 0.5}
\hypersetup{colorlinks=true, citecolor=darkblue, linkcolor=darkblue, urlcolor=darkblue}


\title{Aggie Wine Glass: A By-The-Glass Wine Finder and \\ Recommendation System}

% Authors must not appear in the submitted version. They should be hidden
% as long as the \colmfinalcopy macro remains commented out below.
% Non-anonymous submissions will be rejected without review.

\author{Maya Shah, Ariela Mitrani \& Gabriel Wild\\
Department of Computer Science\\
Texas A\&M University\\
College Station, TX 77840, USA \\
}

% The \author macro works with any number of authors. There are two commands
% used to separate the names and addresses of multiple authors: \And and \AND.
%
% Using \And between authors leaves it to \LaTeX{} to determine where to break
% the lines. Using \AND forces a linebreak at that point. So, if \LaTeX{}
% puts 3 of 4 authors names on the first line, and the last on the second
% line, try using \AND instead of \And before the third author name.

\newcommand{\fix}{\marginpar{FIX}}
\newcommand{\new}{\marginpar{NEW}}

\begin{document}

\ifcolmsubmission
\linenumbers
\fi

\maketitle

\begin{abstract}

Aggie Wine Glass helps Texas A\&M students discover local restaurants and bars that serve wines matching their preferences by the glass. Based on user-inputted taste preferences, pairing requests, and other factors, our recommendation tool provides a curated list of nearby establishments where these wines are available, making wine tasting more accessible and budget-friendly for students. This tool allows users to draw from over 400 by-the-glass wine options in College Station in order to identify places to try new wines without buying bottles. It also helps recommend wines based on existing ratings for the bottles online, and is integrated with LLMs at various steps to give more tailored/specific recommendations.

\end{abstract}

\section{Motivation}

For many college students looking to explore the world of wine tasting, it can be challenging to find local tasting opportunities or try new wines without committing to an entire bottle. Aggie Wine Glass solves this problem by combining a database of wines and their key attributes with an understanding of college student budgets and local availability. Our tool helps Texas A\&M students discover by-the-glass wine options at nearby venues in Bryan-College Station, making wine tasting more accessible, affordable, and convenient. Currently, there are no existing tools that exist to find by-the-glass wine options, so this is fixing a big gap in the market. Similar tools do exist, but not with the specific functionality we provide. Many websites exist to help locate stores to buy wine bottles, and many list important characteristics of the wine such as grape variety, region, and flavor profile. Vivino, for example, is a well-known website and app that allows users to filter wines based on various attributes, leave reviews, and even purchase bottles. However, it does not help users find restaurants or bars where they can try wines by the glass, nor does it allow for filtering top wine lists for specific cities. While our app has some similar features to existing technologies, the main goal is to provide a way for users to find wine tasting recommendations in their area without having to manually scour different restaurant menus. 

\section{Data Collection}

Before we started this project, we identified that building the database of local wines would be the most difficult part of our implementation. In order to have a high-performing system, we needed to compile a sizable number of different wines and many different characteristics of each of these wines. While we searched for an existing wine API to gather information about the wines offered at local restaurants, we did not find a publicly available API we could use, and thus had to create our own web scraper. Creating the database was a two step process: Collecting a list of restaurants, their wines and their prices, and  extracting the characteristics of each wine.

\subsection{Menu Processing}

The first part of the database building process was finding restaurants that had by-the-glass wine offerings, and then processing the information from their menus into a standardized format with the wine name, restaurant name, and glass price. The first obstacle was that not every restaurant in College Station with wine offerings had an online menu. Because of this, we did ground research and physically went to take photos of wine lists for the restaurants that did not have their wine menus posted online. 

From here, we had to extract the relevant information from the menus and put it into a Google Sheet. While this can be done manually by copy-pasting the wine name and the price, this would be incredibly inefficient, especially with over 400 wines from more than 20 different websites. Instead, these menus were copy pasted into ChatGPT, who then outputted the wines and their prices in a table format. This data still had to be manually revised and occasionally changed due to errors in ChatGPT's processing, but using an LLM for this preprocessing step sped up the process exponentially.

An important trend we discovered during this step was that many local restaurants offered the same wine. However, the labeling convention on wine menus differed. In order to make sure that our system recognized these wines as the same wine, we used wine IDs to identify when two wines were the same, regardless of naming convention. The details of this ID and the other wine information follow in the next section.

\subsection{Vivino Data Collection}

The second half of the database building involved getting information about the wine characteristics. For each wine, we extracted the following data points: Bottle Price, Winery, Year, Wine Style, Region, Grape Varieties,	ABV,	Dry/Sweet	 Rating, Tannin Rating, Soft/Acidic Rating, Flavor Profile, Pairings, Rating,	 Category, Light/Bold	 Rating, Profile Specifics, Fizziness, Country, and Wine ID. To get these characteristics, we used Vivino as our base. Vivino is a wine database that has information on a large variety of different wines, as well as user ratings. Using Vivino allowed us to take all of the characteristics from one place, ensuring consistency in these metrics across different wines. 

The information in each of these categories is scattered throughout various places in both the page's HTML and separate JSON files, so our scraping approach required a mix of both types of request. By sending an HTML request to a wine's Vivino page, we can extract most of the characteristics, as well as the wine's ID. The taste characteristics required an additional call to Vivino's taste API, which used the ID in the request URL. Most of the HTML locations for these characteristics were found by manually going through the HTML and locating the regions where the information showed up, as well as monitoring network requests to see which information came from a separate API call.

To use this scraper, we needed the Vivino link for each of the individual wines in the list. Unfortunately, there was not a good way to automate this, especially because some of the wines were challenging to match to Vivino pages, or there were multiple different pages for one type of wine. Therefore, the link finding was done manually. A general page (no year information) was found for the wines that didn't have a vintage specified on the menus, and the ones that did had the link to that specific vintage used.

Not every wine has a complete profile on Vivino. This meant that even after running our scraper on all of the links, there were still gaps in the wine profiles. To have a complete data set, every attribute except for year needed to be filled in. To this end, we used Gemini through Jupyter Notebook to fill in the missing attributes for these wines. However, knowing Gemini isn't perfect, we also did validation on these results. While Gemini was instructed not to change any of the data that was already present, it occasionally would for one of two reasons - the data scraped from Vivino was inaccurate, or Gemini was unfamiliar with the wine. To deal with the first case, all cases where Gemini's output differed from our scraper's were highlighted and manually resolved. For the second case, we passed the entire dataset through ChatGPT and asked it to identify any characteristics that were very atypical and likely wrong, and corrected these as well.

\subsection{Drawbacks and Challenges}

There are a few key drawbacks with our approach. The first involves the data being scraped from Vivino. Initially, we were also getting price from a separate pricing API, but due to issues with Vivino's currency (occasionally returning in other currencies like TWD with no clear conversion to the USD amount), we scrape these from the HTML as well. However, because the most accurate price data comes from the pricing API, the bottle prices we found are not always exact or even matched what Vivino had listed. In addition, even for the data that we scraped accurately every time, Vivino itself is not an objective source. Many users have complained about biased ratings on the site, and the tasting profiles listed are based on users reviews, which can be inaccurate. Also, Vivino's backend can change drastically between wines, making it hard or impossible to accurately scrape the same type of data from different wines.

Another issue with this method is that menus change. Especially in regards to wine selection, restaurants will often change what they are offering based on what is available, and rotate through these rather quickly. However, both of these issues could be resolved if we were to continue working on this project by allowing restaurants that want to participate to input their own data. With something like a restaurant account, every restaurant could put its wines in and make sure their profiles are accurate. However, due to the scope of this project, scraping Vivino and using currently available menus was the best approach. The above is just an example of future work that could be done to mitigate some of these issues.

\section{Submission of conference papers to COLM 2025}

COLM requires electronic submissions, processed by
\url{https://openreview.net/}. See COLM's website for more instructions.
The format for the submissions is a variant of the NeurIPS and ICLR formats.
Please read carefully the instructions below, and follow them
faithfully.


\subsection{Style}

Papers to be submitted to COLM 2025 must be prepared according to the
instructions presented here.

%% Please note that we have introduced automatic line number generation
%% into the style file for \LaTeXe. This is to help reviewers
%% refer to specific lines of the paper when they make their comments. Please do
%% NOT refer to these line numbers in your paper as they will be removed from the
%% style file for the final version of accepted papers.

Authors are required to use the COLM \LaTeX{} style files obtainable at the
COLM website. Please make sure you use the current files and
not previous versions. Tweaking the style files may be grounds for rejection.

\subsubsection{Copy Options}

If your paper is ultimately accepted, the option {\tt
  {\textbackslash}final} should be set  for the {\tt {\textbackslash}usepackage[submission]\{colm2025\_conference\}} command for the camera ready version. The {\tt submission} options is the default, and is to be used for all submissions during the review process. It also turns on the line numbers. If you wish to submit a preprint, the option {\tt preprint} should be used.
  
  

\subsection{Retrieval of style files}

The style files for COLM and other conference information are available online at:
\begin{center}
   \url{http://www.colmweb.org/}
\end{center}
The file \verb+colm2025_conference.pdf+ contains these
instructions and illustrates the
various formatting requirements your COLM paper must satisfy.
Submissions must be made using \LaTeX{} and the style files
\verb+colm2025_conference.sty+ and \verb+colm2025_conference.bst+ (to be used with \LaTeX{}2e). The file
\verb+colm2025_conference.tex+ may be used as a ``shell'' for writing your paper. All you
have to do is replace the author, title, abstract, and text of the paper with
your own.

The formatting instructions contained in these style files are summarized in
sections \ref{gen_inst}, \ref{headings}, and \ref{others} below.

\section{General formatting instructions}
\label{gen_inst}

The text must be confined within a rectangle 5.5~inches (33~picas) wide and
9~inches (54~picas) long. The left margin is 1.5~inch (9~picas).
Use 10~point type with a vertical spacing of 11~points. Palatino is the
preferred typeface throughout, and is mandatory for the main text. Paragraphs are separated by 1/2~line space, with no indentation. 

Paper title is 17~point and left-aligned.
All pages should start at 1~inch (6~picas) from the top of the page.

Please verify that any custom header information you may add does not override the style defined in this document. This has been known to occur especially when submissions are converted to a new template from a previous one (i.e., for re-submission to a different venue). 

Authors' names are
set in boldface, and each name is placed above its corresponding
address. The lead author's name is to be listed first, and
the co-authors' names are set to follow. Authors sharing the
same address can be on the same line.

Please pay special attention to the instructions in section \ref{others}
regarding figures, tables, acknowledgements, and references.


There will be a strict upper limit of 9 pages for the main text of the initial submission, with unlimited additional pages for citations. 

We strongly recommend following arXiv's guidelines for making your paper friendly for HTML conversion: \url{https://info.arxiv.org/help/submit_latex_best_practices.html}.


\section{Headings: first level}
\label{headings}

First level headings are in lower case (except for first word and proper nouns), bold face,
flush left and in point size 12. One line space before the first level
heading and 1/2~line space after the first level heading.

\subsection{Headings: second level}

Second level headings are in lower case (except for first word and proper nouns), bold face,
flush left and in point size 10. One line space before the second level
heading and 1/2~line space after the second level heading.

\subsubsection{Headings: third level}

Third level headings are in lower case (except for first word and proper nouns), bold face, italics, 
flush left and in point size 10. One line space before the third level
heading and 1/2~line space after the third level heading.

\section{Citations, figures, tables, references}\label{others}

These instructions apply to everyone, regardless of the formatter being used.

\subsection{Citations within the text}

Citations within the text should be based on the \texttt{natbib} package
and include the authors' last names and year (with the ``et~al.'' construct
for more than two authors). When the authors or the publication are
included in the sentence, the citation should not be in parenthesis using \verb|\citet{}| (as
in ``See \citet{Vaswani+2017} for more information.''). Otherwise, the citation
should be in parenthesis using \verb|\citep{}| (as in ``Transformers are a key tool
for developing language models~\citep{Vaswani+2017}.'').

The corresponding references are to be listed in alphabetical order of
authors, in the \textsc{References} section. As to the format of the
references themselves, any style is acceptable as long as it is used
consistently.

\subsection{Footnotes}

Indicate footnotes with a number\footnote{Sample of the first footnote} in the
text. Place the footnotes at the bottom of the page on which they appear.
Precede the footnote with a horizontal rule of 2~inches
(12~picas).\footnote{Sample of the second footnote}

\subsection{Figures}

All artwork must be neat, clean, and legible. Lines should be dark
enough for purposes of reproduction; art work should not be
hand-drawn. Any text within the figure must be readable. We ask to not use font sizes below {\tt small}. We strongly recommend to use vector representations (e.g., pdf or svg) for all diagrams. 
We strongly recommend positioning all figures at the top or bottom of the page.

The figure number and caption always appear below the figure. Place one line space before the figure caption, and one line space after the figure. The figure caption is lower case (except for first word and proper nouns); figures are numbered consecutively.
Make sure the figure caption does not get separated from the figure.
Leave sufficient space to avoid splitting the figure and figure caption.

You may use color figures.
However, it is best for the
figure captions and the paper body to make sense if the paper is printed
either in black/white or in color.
\begin{figure}[t]
\begin{center}
%\framebox[4.0in]{$\;$}
\fbox{\rule[-.5cm]{0cm}{4cm} \rule[-.5cm]{4cm}{0cm}}
\end{center}
\caption{Sample figure caption.}
\end{figure}

\subsection{Tables}

All tables must be centered, neat, clean and legible. Do not use hand-drawn tables. The table number and title always appear below the table. See Table~\ref{sample-table}. Please do not use font sizes below {\tt small} in tables. We recommend using {\tt booktabs} or a similar package to style tables. 
We strongly recommend positioning all tables at the top or bottom of the page.

Place one line space before the table title, one line space after the table title, and one line space after the table. The table title must be lowercase (except for first word and proper nouns); tables are numbered consecutively.

\begin{table}[t]
\begin{center}
\begin{tabular}{ll}
\toprule
\multicolumn{1}{c}{\bf PART}  &\multicolumn{1}{c}{\bf DESCRIPTION} \\
\midrule
Dendrite         &Input terminal \\
Axon             &Output terminal \\
Soma             &Cell body (contains cell nucleus) \\
\bottomrule
\end{tabular}
\end{center}
\caption{Sample table title}\label{sample-table}
\end{table}




\section{Final instructions}
Do not change any aspects of the formatting parameters in the style files.
In particular, do not modify the width or length of the rectangle the text
should fit into, and do not change font sizes (except perhaps in the
\textsc{References} section; see below). Please note that pages should be
numbered.

\section{Preparing PostScript or PDF files}

Please prepare PostScript or PDF files with paper size ``US Letter'', and
not, for example, ``A4''. The -t
letter option on dvips will produce US Letter files.

Consider directly generating PDF files using \verb+pdflatex+
(especially if you are a MiKTeX user).
PDF figures must be substituted for EPS figures, however.

Otherwise, please generate your PostScript and PDF files with the following commands:
\begin{verbatim}
dvips mypaper.dvi -t letter -Ppdf -G0 -o mypaper.ps
ps2pdf mypaper.ps mypaper.pdf
\end{verbatim}

\subsection{Margins in LaTeX}

Most of the margin problems come from figures positioned by hand using
\verb+\special+ or other commands. We suggest using the command
\verb+\includegraphics+
from the graphicx package. Always specify the figure width as a multiple of
the line width as in the example below using .eps graphics
\begin{verbatim}
   \usepackage[dvips]{graphicx} ...
   \includegraphics[width=0.8\linewidth]{myfile.eps}
\end{verbatim}
or % Apr 2009 addition
\begin{verbatim}
   \usepackage[pdftex]{graphicx} ...
   \includegraphics[width=0.8\linewidth]{myfile.pdf}
\end{verbatim}
for .pdf graphics.
See section~4.4 in the graphics bundle documentation (\url{http://www.ctan.org/tex-archive/macros/latex/required/graphics/grfguide.ps})

A number of width problems arise when LaTeX cannot properly hyphenate a
line. Please give LaTeX hyphenation hints using the \verb+\-+ command.

\section*{Author Contributions}
If you'd like to, you may include  a section for author contributions as is done
in many journals. This is optional and at the discretion of the authors.

\section*{Acknowledgments}
Use unnumbered first level headings for the acknowledgments. All
acknowledgments, including those to funding agencies, go at the end of the paper.

\section*{Ethics Statement}
Authors can add an optional ethics statement to the paper. 
For papers that touch on ethical issues, this section will be evaluated as part of the review process. The ethics statement should come at the end of the paper. It does not count toward the page limit, but should not be more than 1 page. 



\bibliography{colm2025_conference}
\bibliographystyle{colm2025_conference}

\appendix
\section{Appendix}
You may include other additional sections here.

\end{document}
